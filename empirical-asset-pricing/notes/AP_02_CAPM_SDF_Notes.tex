%!TEX program = xelatex
\documentclass[11pt]{article}

\usepackage[margin=1in]{geometry}
\usepackage{amsmath, amssymb, amsthm}
\usepackage{mathtools}
\usepackage{booktabs}
\usepackage{enumitem}
\usepackage{hyperref}

\setlength{\parskip}{6pt}
\setlength{\parindent}{0pt}

\newcommand{\R}{\mathbb{R}}
\newcommand{\E}{\mathbb{E}}
\newcommand{\Var}{\mathrm{Var}}
\newcommand{\Cov}{\mathrm{Cov}}

\title{AP-02 Notes: CAPM and SDF Foundations}
\author{Dr. Ian Helfrich}
\date{\today}

\begin{document}
\maketitle

\section*{Purpose of These Notes}
These notes are the ``book version'' of AP-02. The slides are the live walkthrough.
Here I write the CAPM derivation twice and keep the logic tight.

\section*{Roadmap}
\begin{enumerate}[leftmargin=18pt]
\item Derive CAPM from mean-variance optimization.
\item Re-derive CAPM from the SDF pricing equation.
\item Connect to zero-beta CAPM and factor models.
\end{enumerate}

\section*{1. Mean-Variance Setup}
Let $R \in \R^N$ be risky returns with mean $\mu$ and covariance $\Sigma$.
A risk-free asset yields gross return $R_f$.
A portfolio with risky weights $w$ has return
\[
R_p = R_f + w'(R - R_f\mathbf{1}).
\]
Then
\begin{align*}
\E[R_p] &= R_f + w'(\mu - R_f\mathbf{1}), \\
\Var(R_p) &= w'\Sigma w.
\end{align*}

\section*{2. Mean-Variance Optimization}
A mean-variance investor solves
\[
\max_w \; \E[R_p] - \frac{\gamma}{2}\Var(R_p).
\]
The first-order condition is
\[
\mu - R_f\mathbf{1} - \gamma\Sigma w = 0.
\]
So optimal risky holdings satisfy
\[
 w \propto \Sigma^{-1}(\mu - R_f\mathbf{1}).
\]
All investors hold the same risky portfolio up to scale.

\section*{3. Market Clearing and the CAPM}
In equilibrium, the market portfolio $M$ must be mean-variance efficient.
Let $w_M$ denote market weights. Then for some scalar $\kappa$,
\[
\Sigma w_M = \kappa(\mu - R_f\mathbf{1}).
\]
Take the $i$th element:
\[
(\Sigma w_M)_i = \kappa(\mu_i - R_f).
\]
But $(\Sigma w_M)_i = \Cov(R_i, R_M)$, so
\[
\mu_i - R_f = \frac{1}{\kappa}\Cov(R_i, R_M).
\]
Apply the same equation to the market itself:
\[
\mu_M - R_f = \frac{1}{\kappa}\Var(R_M).
\]
Divide to obtain the CAPM:
\[
\E[R_i] - R_f = \beta_i(\E[R_M] - R_f),
\quad \beta_i = \frac{\Cov(R_i, R_M)}{\Var(R_M)}.
\]

\section*{4. Geometric Interpretation}
The efficient frontier with a risk-free asset becomes a straight line (the capital market line).
The tangency portfolio on the risky frontier is the market portfolio.
CAPM is the security market line: expected returns are linear in betas.

\section*{5. SDF Derivation}
The stochastic discount factor (SDF) $m$ satisfies
\[
\E[m R_i] = 1
\]
for every asset $i$.
Assume a linear SDF:
\[
 m = a - b R_M.
\]
Use covariance decomposition:
\begin{align*}
\E[m R_i] &= \E[m]\E[R_i] + \Cov(m, R_i) = 1, \\
\E[R_i] - R_f &= -\frac{\Cov(m, R_i)}{\E[m]}.
\end{align*}
Since $\Cov(m, R_i) = -b\Cov(R_M, R_i)$,
\[
\E[R_i] - R_f = \frac{b}{\E[m]}\Cov(R_i, R_M).
\]
Apply the same equation to $R_M$ and divide to recover CAPM.

\section*{6. Zero-Beta CAPM}
If no risk-free asset exists, CAPM becomes
\[
\E[R_i] = \E[R_Z] + \beta_i^Z(\E[R_M] - \E[R_Z]),
\]
where $R_Z$ is the return on the zero-beta portfolio.
This is a common qualifier extension.

\section*{7. From CAPM to Factor Models}
CAPM is a one-factor model. Empirical work often uses
\[
\E[R_i] = R_f + \beta_{i1}\lambda_1 + \cdots + \beta_{ik}\lambda_k.
\]
CAPM is the special case $k=1$ with the market factor.

\section*{8. Worked Examples}
\textbf{Example 1 (Beta and Expected Return).}
Let $R_f = 1.02$, $\E[R_M] = 1.10$, $\Var(R_M) = 0.04$, and $\Cov(R_i,R_M) = 0.06$.
Then
\[
\beta_i = \frac{0.06}{0.04} = 1.5,
\quad \E[R_i] = 1.02 + 1.5(1.10 - 1.02) = 1.14.
\]

\textbf{Example 2 (From Covariance Matrix).}
Let
\[
\Sigma = \begin{bmatrix}0.04 & 0.01 \\ 0.01 & 0.09\end{bmatrix},
\quad w_M = \begin{bmatrix}0.6 \\ 0.4\end{bmatrix}.
\]
Then
\[
\Var(R_M) = w_M'\Sigma w_M = 0.0432,
\quad \Cov(R_1,R_M) = 0.028,
\]
so $\beta_1 \approx 0.648$.

\section*{9. Common Pitfalls}
\begin{itemize}[leftmargin=18pt]
\item Confusing gross returns with net returns.
\item Forgetting that the market portfolio must be efficient.
\item Dropping the covariance identity $(\Sigma w_M)_i = \Cov(R_i, R_M)$.
\item Mixing up correlation with beta.
\end{itemize}

\section*{10. Exam Checklist}
\begin{enumerate}[leftmargin=18pt]
\item Write the mean-variance FOC and solve for $w$.
\item Use market clearing to identify $w_M$.
\item Convert to the covariance form and solve for $\beta_i$.
\item Derive CAPM again from the SDF pricing equation.
\item State the zero-beta CAPM extension.
\end{enumerate}

\section*{11. Practice Problems}
\begin{enumerate}[leftmargin=18pt]
\item Prove CAPM using only the covariance identity and market efficiency.
\item Derive $a$ and $b$ for the linear SDF from the risk-free asset and market equations.
\item Explain how CAPM nests inside a two-factor model and how you would test it.
\end{enumerate}

\section*{Instructor Note}
I am an independent researcher and PhD (Georgia Institute of Technology), not currently affiliated
with any institution. These notes are my own work.

\end{document}
