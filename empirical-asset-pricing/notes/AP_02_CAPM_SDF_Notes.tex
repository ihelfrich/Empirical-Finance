%!TEX program = xelatex
\documentclass[11pt]{article}

\usepackage[margin=1in]{geometry}
\usepackage{amsmath, amssymb, amsthm}
\usepackage{mathtools}
\usepackage{booktabs}
\usepackage{enumitem}
\usepackage{hyperref}

\setlength{\parskip}{6pt}
\setlength{\parindent}{0pt}

\newcommand{\R}{\mathbb{R}}
\newcommand{\E}{\mathbb{E}}
\newcommand{\Var}{\mathrm{Var}}
\newcommand{\Cov}{\mathrm{Cov}}

\newtheorem{definition}{Definition}
\newtheorem{lemma}{Lemma}
\newtheorem{theorem}{Theorem}

\title{AP-02 Notes: CAPM and SDF Foundations}
\author{Dr. Ian Helfrich}
\date{\today}

\begin{document}
\maketitle
\tableofcontents
\newpage

\section{Purpose and Style}
These notes are the ``book version'' of AP-02. The slides are the live walkthrough.
Here I derive the CAPM twice and keep the logic tight.

\section{Mean-Variance Setup}
Let $R \in \R^N$ be risky returns with mean $\mu$ and covariance $\Sigma$.
A risk-free asset yields gross return $R_f$. A portfolio with risky weights $w$ has
\[
R_p = R_f + w'(R - R_f\mathbf{1}).
\]
Then
\begin{align*}
\E[R_p] &= R_f + w'(\mu - R_f\mathbf{1}), \\
\Var(R_p) &= w'\Sigma w.
\end{align*}

\section{Mean-Variance Optimization}
A mean-variance investor solves
\[
\max_w \; \E[R_p] - \frac{\gamma}{2}\Var(R_p).
\]
The first-order condition is
\[
\mu - R_f\mathbf{1} - \gamma\Sigma w = 0.
\]
So optimal risky holdings satisfy
\[
 w \propto \Sigma^{-1}(\mu - R_f\mathbf{1}).
\]
All investors hold the same risky portfolio up to scale.

\section{Market Clearing and the CAPM}
In equilibrium, the market portfolio $M$ must be mean-variance efficient.
Let $w_M$ denote market weights. Then for some scalar $\kappa$,
\[
\Sigma w_M = \kappa(\mu - R_f\mathbf{1}).
\]
Take the $i$th element:
\[
(\Sigma w_M)_i = \kappa(\mu_i - R_f).
\]
But $(\Sigma w_M)_i = \Cov(R_i, R_M)$, so
\[
\mu_i - R_f = \frac{1}{\kappa}\Cov(R_i, R_M).
\]
Apply the same equation to the market itself:
\[
\mu_M - R_f = \frac{1}{\kappa}\Var(R_M).
\]
Divide to obtain the CAPM:
\[
\E[R_i] - R_f = \beta_i(\E[R_M] - R_f),
\quad \beta_i = \frac{\Cov(R_i, R_M)}{\Var(R_M)}.
\]

\begin{theorem}[CAPM from mean-variance]
If a risk-free asset exists and investors are mean-variance optimizers, then
\[
\E[R_i] - R_f = \beta_i(\E[R_M] - R_f).
\]
\end{theorem}
\begin{proof}
The steps above show that $\Sigma w_M$ is proportional to $\mu - R_f\mathbf{1}$.
Convert to covariance form, apply the equation to the market, and divide.
\end{proof}

\section{Geometric Interpretation}
The efficient frontier with a risk-free asset becomes a straight line (the capital market line).
The tangency portfolio on the risky frontier is the market portfolio.
CAPM is the security market line: expected returns are linear in betas.

\section{SDF Derivation}
\begin{definition}[Stochastic discount factor]
A stochastic discount factor (SDF) $m$ satisfies
\[
\E[m R_i] = 1
\]
for every asset $i$.
\end{definition}

Assume a linear SDF:
\[
 m = a - b R_M.
\]
Use covariance decomposition:
\begin{align*}
\E[m R_i] &= \E[m]\E[R_i] + \Cov(m, R_i) = 1, \\
\E[R_i] - R_f &= -\frac{\Cov(m, R_i)}{\E[m]}.
\end{align*}
Since $\Cov(m, R_i) = -b\Cov(R_M, R_i)$,
\[
\E[R_i] - R_f = \frac{b}{\E[m]}\Cov(R_i, R_M).
\]
Apply the same equation to $R_M$ and divide to recover CAPM.

\begin{theorem}[CAPM from the SDF]
If the SDF is linear in the market return, then CAPM holds.
\end{theorem}
\begin{proof}
Use the covariance decomposition above and divide by the market equation.
\end{proof}

\section{Solving for $a$ and $b$ in the Linear SDF}
The two key equations are
\[
R_f \E[m] = 1, \quad \E[m R_M] = 1.
\]
With $m = a - b R_M$, this becomes
\begin{align*}
R_f(a - b\E[R_M]) &= 1, \\
 a\E[R_M] - b\E[R_M^2] &= 1.
\end{align*}
This is a two-by-two linear system. Solve for $(a,b)$ and substitute back into the pricing equation.

\section{Zero-Beta CAPM}
If no risk-free asset exists, CAPM becomes
\[
\E[R_i] = \E[R_Z] + \beta_i^Z(\E[R_M] - \E[R_Z]),
\]
where $R_Z$ is the return on the zero-beta portfolio.

\section{From CAPM to Factor Models}
CAPM is a one-factor model. Empirical work often uses
\[
\E[R_i] = R_f + \beta_{i1}\lambda_1 + \cdots + \beta_{ik}\lambda_k.
\]
CAPM is the special case $k=1$ with the market factor.

\section{Cross-Sectional Regression Form}
If CAPM holds, then for each asset $i$,
\[
\E[R_i] - R_f = \beta_i \lambda_M,
\quad \lambda_M = \E[R_M] - R_f.
\]
Empirically, this becomes a cross-sectional regression of average excess returns on betas.
This is the link to Fama-MacBeth.

\section{Time-Series Beta Estimation}
In practice, betas are estimated from time-series regressions:
\[
R_{i,t} - R_{f,t} = \alpha_i + \beta_i (R_{M,t} - R_{f,t}) + \varepsilon_{i,t}.
\]
CAPM predicts $\alpha_i = 0$ for all assets. This is a central empirical test.

\section{Pricing Errors and Alphas}
If CAPM fails, then expected returns satisfy
\[
\E[R_i] - R_f = \beta_i \lambda_M + \alpha_i,
\]
where $\alpha_i$ is a pricing error. In empirical work, persistent nonzero $\alpha_i$
signal model misspecification or measurement issues.

\section{Tangency Portfolio via Lagrange Multipliers}
Solve the tangency portfolio directly. Let $w$ be risky weights and impose $w'\mathbf{1}=1$.
Maximize the Sharpe ratio or equivalently
\[
\max_w \\; w'(\mu - R_f\mathbf{1}) - \frac{\gamma}{2} w'\Sigma w
\quad \text{s.t. } w'\mathbf{1}=1.
\]
The Lagrangian is
\[
\mathcal{L}(w, \lambda) = w'(\mu - R_f\mathbf{1}) - \frac{\gamma}{2} w'\Sigma w + \lambda(1 - w'\mathbf{1}).
\]
FOC:
\[
\mu - R_f\mathbf{1} - \gamma\Sigma w - \lambda\mathbf{1} = 0.
\]
Solve for $w$ and normalize. This yields the tangency portfolio and the capital market line.

\section{Two-Fund Separation}
Because all investors hold the same tangency portfolio up to scale,
any mean-variance investor holds a combination of the risk-free asset and $M$.
This is two-fund separation and is central to the CAPM logic.

\section{Security Market Line}
CAPM implies a linear relationship between expected excess returns and beta:
\[
\E[R_i] - R_f = \beta_i(\E[R_M]-R_f).
\]
This line is the security market line. The slope is the market risk premium.
Assets above the line have positive alpha; assets below the line have negative alpha.

\section{Hansen-Jagannathan Bound (Optional)}
The SDF $m$ implies a bound on Sharpe ratios:
\[
\frac{\E[R_p] - R_f}{\sqrt{\Var(R_p)}} \le \frac{\sqrt{\Var(m)}}{\E[m]}.
\]
This connects asset pricing restrictions to volatility of the discount factor.
You can view CAPM as a specific SDF that enforces a single-factor bound.

\section{Qualifier-Style Problem A (Full Walkthrough)}
\textbf{Problem.} Assume mean-variance preferences and a risk-free asset. Derive the CAPM
and interpret $\beta_i$.

\subsection*{Solution}
Step 1. Write the mean-variance objective and FOC:
\[
\mu - R_f\mathbf{1} - \gamma\Sigma w = 0.
\]
Step 2. Identify the market portfolio by market clearing and write
\[
\Sigma w_M = \kappa(\mu - R_f\mathbf{1}).
\]
Step 3. Convert to covariance form: $(\Sigma w_M)_i = \Cov(R_i,R_M)$.
Step 4. Solve for $\kappa$ using the market itself and obtain CAPM.

\section{Qualifier-Style Problem B (SDF)}
\textbf{Problem.} Suppose prices satisfy $\E[mR_i]=1$ and $m=a-bR_M$. Prove the CAPM.

\subsection*{Solution}
Use covariance decomposition:
\[
\E[R_i]-R_f = -\frac{\Cov(m,R_i)}{\E[m]}.
\]
Substitute $m=a-bR_M$ to get
\[
\E[R_i]-R_f = \frac{b}{\E[m]}\Cov(R_i,R_M).
\]
Apply the same equation to $R_M$ and divide to obtain CAPM.

\section{Qualifier-Style Problem C (Empirical Test)}
\textbf{Problem.} You estimate time-series betas for 25 portfolios. You then run a
cross-sectional regression of average excess returns on betas and obtain a large intercept.
Interpret the result in the context of CAPM.

\subsection*{Solution}
CAPM predicts zero intercept and slope equal to the market risk premium. A large intercept
signals pricing errors or a missing factor. It can also arise from measurement error in betas
or a flawed market proxy. The correct qualifier answer ties the intercept to model failure.

\section{Worked Examples}
\subsection*{Example 1: Beta and Expected Return}
Let $R_f = 1.02$, $\E[R_M] = 1.10$, $\Var(R_M) = 0.04$, and $\Cov(R_i,R_M) = 0.06$.
Then
\[
\beta_i = \frac{0.06}{0.04} = 1.5,
\quad \E[R_i] = 1.02 + 1.5(1.10 - 1.02) = 1.14.
\]

\subsection*{Example 2: From Covariance Matrix}
Let
\[
\Sigma = \begin{bmatrix}0.04 & 0.01 \\ 0.01 & 0.09\end{bmatrix},
\quad w_M = \begin{bmatrix}0.6 \\ 0.4\end{bmatrix}.
\]
Then
\[
\Var(R_M) = w_M'\Sigma w_M = 0.0432,
\quad \Cov(R_1,R_M) = 0.028,
\]
so $\beta_1 \approx 0.648$.

\subsection*{Example 3: SDF Parameters}
Let $R_f = 1.01$, $\E[R_M]=1.08$, and $\E[R_M^2]=1.20$.
Solve
$R_f(a-b\E[R_M])=1$ and $a\E[R_M]-b\E[R_M^2]=1$ for $(a,b)$.
Then compute $m=a-bR_M$.

\subsection*{Example 4: Alpha Check}
Suppose an asset has $\beta_i = 0.8$ and average excess return $0.06$.
If the market risk premium is $0.08$, CAPM predicts $0.8 \\times 0.08 = 0.064$.
The pricing error is $\alpha_i = 0.06 - 0.064 = -0.004$.
In words: this asset earns too little for its beta.

\section{Step-by-Step Blueprint (Exam Edition)}
When time is tight, follow this exact structure:
\begin{enumerate}[leftmargin=18pt]
\item Write the mean-variance objective and the FOC.
\item State that the market portfolio is the tangency portfolio.
\item Use $\Sigma w_M = \kappa(\mu - R_f\mathbf{1})$.
\item Convert to covariance form and solve for $\kappa$ using the market.
\item Conclude CAPM and interpret $\beta_i$.
\item Re-derive from the SDF in two lines as a check.
\end{enumerate}

\section{Common Pitfalls}
\begin{itemize}[leftmargin=18pt]
\item Confusing gross returns with net returns.
\item Forgetting that the market portfolio must be efficient.
\item Dropping the covariance identity $(\Sigma w_M)_i = \Cov(R_i, R_M)$.
\item Mixing up correlation with beta.
\item Treating the CAPM as an empirical truth instead of an equilibrium benchmark.
\end{itemize}

\section{Exam Checklist}
\begin{enumerate}[leftmargin=18pt]
\item Write the mean-variance FOC and solve for $w$.
\item Use market clearing to identify $w_M$.
\item Convert to covariance form and solve for $\beta_i$.
\item Derive CAPM again from the SDF pricing equation.
\item State the zero-beta CAPM extension.
\end{enumerate}

\section{Practice Problems}
\begin{enumerate}[leftmargin=18pt]
\item Prove CAPM using only the covariance identity and market efficiency.
\item Derive $a$ and $b$ for the linear SDF from the risk-free asset and market equations.
\item Explain how CAPM nests inside a two-factor model and how you would test it.
\item Derive the zero-beta CAPM from a mean-variance frontier without $R_f$.
\item Use the SDF to show why idiosyncratic risk is not priced.
\item Show how CAPM changes if the market proxy is measured with error.
\item Derive the tangency portfolio weights explicitly in a two-asset example.
\item Use the SDF to compute the price of a risk-free asset and verify $R_f = 1/\E[m]$.
\end{enumerate}

\section{Solutions}
\subsection*{Solution 1}
Use $\Sigma w_M = \kappa(\mu - R_f\mathbf{1})$, take element $i$, and apply the same equation to $M$.
Divide to obtain CAPM.

\subsection*{Solution 2}
Solve the two equations
$R_f\E[m] = 1$ and $\E[mR_M]=1$ with $m=a-bR_M$. Substitute
$\E[m]=a-b\E[R_M]$ and $\E[mR_M]=a\E[R_M]-b\E[R_M^2]$ and solve for $(a,b)$.

\subsection*{Solution 3}
A two-factor model is $\E[R_i] = R_f + \beta_{i1}\lambda_1 + \beta_{i2}\lambda_2$.
CAPM is nested by restricting $\lambda_2=0$. Test with a cross-sectional regression or
GMM on pricing errors.

\subsection*{Solution 4}
Without $R_f$, the efficient frontier does not intersect the risk-free point.
The tangent line is replaced by a line through the zero-beta portfolio $Z$ and $M$.
Thus $\E[R_i] = \E[R_Z] + \beta_i^Z(\E[R_M]-\E[R_Z])$.

\subsection*{Solution 5}
From $\E[mR_i]=1$, we have
$\E[R_i]-R_f = -\Cov(m,R_i)/\E[m]$.
If $R_i$ has only idiosyncratic risk, then $\Cov(m,R_i)=0$ and the premium is zero.

\subsection*{Solution 6}
Measurement error in $R_M$ biases estimated betas toward zero and can create spurious
pricing errors. This is a standard empirical pitfall.

\subsection*{Solution 7}
With two assets, write $\Sigma$ and $\mu$. Solve
$w \\propto \\Sigma^{-1}(\\mu - R_f\\mathbf{1})$ and normalize to sum to one.
This is the tangency portfolio.

\subsection*{Solution 8}
From $\\E[mR_f]=1$ with constant $R_f$, we get $R_f\\E[m]=1$ and thus
$R_f = 1/\\E[m]$.

\section*{Instructor Note}
I am an independent researcher and PhD (Georgia Institute of Technology), not currently affiliated
with any institution. These notes are my own work.

\end{document}
