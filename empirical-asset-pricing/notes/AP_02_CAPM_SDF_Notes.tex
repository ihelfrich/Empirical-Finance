%!TEX program = xelatex
\documentclass[11pt]{article}

\usepackage[margin=1in]{geometry}
\usepackage{amsmath, amssymb, amsthm}
\usepackage{mathtools}
\usepackage{booktabs}
\usepackage{enumitem}
\usepackage{hyperref}

\setlength{\parskip}{6pt}
\setlength{\parindent}{0pt}

\newcommand{\R}{\mathbb{R}}
\newcommand{\E}{\mathbb{E}}
\newcommand{\Var}{\mathrm{Var}}
\newcommand{\Cov}{\mathrm{Cov}}

\title{AP-02 Notes: CAPM and SDF Foundations}
\author{Dr. Ian Helfrich}
\date{\today}

\begin{document}
\maketitle

\section*{Purpose of These Notes}
These notes are the ``book version'' of AP-02. The slides are the live exam walkthrough.
This document is the full derivation with commentary and structure.

\section*{Mean-Variance Setup}
Consider $N$ risky returns $R \in \R^N$ with mean $\mu$ and covariance $\Sigma$.
A risk-free asset yields gross return $R_f$.
A portfolio with risky weights $w$ has return
\[
R_p = R_f + w'(R - R_f\mathbf{1}).
\]
Then
\begin{align*}
\E[R_p] &= R_f + w'(\mu - R_f\mathbf{1}), \\
\Var(R_p) &= w'\Sigma w.
\end{align*}

\section*{Mean-Variance Optimization}
A mean-variance investor solves
\[
\max_w \; \E[R_p] - \frac{\gamma}{2}\Var(R_p).
\]
The first-order condition is
\[
\mu - R_f\mathbf{1} - \gamma\Sigma w = 0.
\]
So optimal risky holdings satisfy
\[
 w \propto \Sigma^{-1}(\mu - R_f\mathbf{1}).
\]
All investors hold the same risky portfolio up to scale.

\section*{Market Clearing and the CAPM}
In equilibrium, the market portfolio $M$ must be mean-variance efficient.
Let $w_M$ denote market weights. Then for some scalar $\kappa$,
\[
\Sigma w_M = \kappa(\mu - R_f\mathbf{1}).
\]
Take the $i$th element:
\[
(\Sigma w_M)_i = \kappa(\mu_i - R_f).
\]
But $(\Sigma w_M)_i = \Cov(R_i, R_M)$, so
\[
\mu_i - R_f = \frac{1}{\kappa}\Cov(R_i, R_M).
\]
Apply the same equation to the market itself:
\[
\mu_M - R_f = \frac{1}{\kappa}\Var(R_M).
\]
Divide to obtain the CAPM:
\[
\E[R_i] - R_f = \beta_i(\E[R_M] - R_f),
\quad \beta_i = \frac{\Cov(R_i, R_M)}{\Var(R_M)}.
\]

\section*{SDF Derivation}
The stochastic discount factor (SDF) $m$ satisfies
\[
\E[m R_i] = 1
\]
for every asset $i$.
Assume a linear SDF:
\[
 m = a - b R_M.
\]
Use covariance decomposition:
\begin{align*}
\E[m R_i] &= \E[m]\E[R_i] + \Cov(m, R_i) = 1, \\
\E[R_i] - R_f &= -\frac{\Cov(m, R_i)}{\E[m]}.
\end{align*}
Since $\Cov(m, R_i) = -b\Cov(R_M, R_i)$,
\[
\E[R_i] - R_f = \frac{b}{\E[m]}\Cov(R_i, R_M).
\]
Apply the same equation to $R_M$ and divide to recover CAPM.

\section*{Worked Example}
Let $R_f = 1.02$, $\E[R_M] = 1.10$, $\Var(R_M) = 0.04$, and $\Cov(R_i,R_M) = 0.06$.
Then
\[
\beta_i = \frac{0.06}{0.04} = 1.5
\]
and
\[
\E[R_i] = 1.02 + 1.5(1.10 - 1.02) = 1.14.
\]

\section*{Common Pitfalls}
\begin{itemize}[leftmargin=18pt]
\item Confusing gross returns with net returns.
\item Forgetting the market portfolio must be efficient.
\item Dropping the covariance identity $(\Sigma w_M)_i = \Cov(R_i, R_M)$.
\item Mixing up correlation and beta.
\end{itemize}

\section*{Exam Checklist}
\begin{enumerate}[leftmargin=18pt]
\item Write the mean-variance FOC and solve for $w$.
\item Use market clearing to identify $w_M$.
\item Convert to the covariance form and solve for $\beta_i$.
\item Derive CAPM again from the SDF pricing equation.
\end{enumerate}

\section*{Instructor Note}
I am an independent researcher and PhD (Georgia Institute of Technology), not currently affiliated
with any institution. These notes are my own work.

\end{document}
