%!TEX program = xelatex
\documentclass[11pt]{article}

\usepackage[margin=1in]{geometry}
\usepackage{amsmath, amssymb, amsthm}
\usepackage{mathtools}
\usepackage{booktabs}
\usepackage{enumitem}
\usepackage{hyperref}

\setlength{\parskip}{6pt}
\setlength{\parindent}{0pt}

\newcommand{\R}{\mathbb{R}}
\newcommand{\E}{\mathbb{E}}
\newcommand{\Var}{\mathrm{Var}}
\newcommand{\Cov}{\mathrm{Cov}}

\title{AP-01 Notes: CARA-Normal Equilibrium}
\author{Dr. Ian Helfrich}
\date{\today}

\begin{document}
\maketitle

\section*{Purpose of These Notes}
These notes are the ``book version'' of the AP-01 slides. The slides are the live, exam-style
walkthrough. This document is the slow, careful write-up: every definition, every equation,
and every step I expect you to reproduce under pressure.

\section*{Setup}
We study a one-period economy with $N$ risky payoffs collected in the vector $X$.
Assume
\[
X \sim \mathcal{N}(\mu, \Sigma).
\]
Agents have CARA utility
\[
U(c) = -\exp(-\alpha c), \quad \alpha > 0.
\]
Let $p \in \R^N$ be the price vector of the risky payoffs, and let $\theta \in \R^N$ denote
a portfolio of risky payoffs.

Final wealth can be written as
\[
c = w_0 - p'\theta + \theta'X.
\]

\section*{Certainty Equivalent Under CARA + Normal}
The key move is that CARA utility and normality turn expected utility into a certainty equivalent.
If $c$ is normal, then
\[
\E[U(c)] = -\exp\left(-\alpha\E[c] + \frac{\alpha^2}{2}\Var(c)\right).
\]
Thus maximizing expected utility is equivalent to maximizing the certainty equivalent
\[
CE = \E[c] - \frac{\alpha}{2}\Var(c).
\]
This is the algebraic bridge to mean-variance reasoning.

\section*{Individual Demand}
Compute the mean and variance of $c$:
\begin{align*}
\E[c] &= w_0 - p'\theta + \theta'\mu, \\
\Var(c) &= \theta'\Sigma\theta.
\end{align*}
So the certainty equivalent is
\[
CE = w_0 - p'\theta + \theta'\mu - \frac{\alpha}{2}\theta'\Sigma\theta.
\]
Differentiate with respect to $\theta$ and set equal to zero:
\[
- p + \mu - \alpha\Sigma\theta = 0.
\]
Solve for optimal demand:
\[
\theta^* = \frac{1}{\alpha}\Sigma^{-1}(\mu - p).
\]

\section*{Aggregate Risk Tolerance}
Let agent $h$ have risk aversion $\alpha_h$. Summing demands gives
\[
\sum_h \theta_h^* = \left(\sum_h \frac{1}{\alpha_h}\right)\Sigma^{-1}(\mu - p).
\]
Define total risk tolerance
\[
T = \sum_h \frac{1}{\alpha_h}.
\]
Then aggregate demand is
\[
\sum_h \theta_h^* = T\Sigma^{-1}(\mu - p).
\]

\section*{Market Clearing and Equilibrium Price}
Let $\bar{\theta}$ be the supply vector of risky payoffs. Market clearing requires
\[
\bar{\theta} = T\Sigma^{-1}(\mu - p).
\]
Solve for $p$:
\[
 p = \mu - \frac{1}{T}\Sigma\bar{\theta}.
\]
This is the equilibrium price formula.

\section*{Risk Premia and Interpretation}
Rewrite the price equation as
\[
\mu - p = \frac{1}{T}\Sigma\bar{\theta}.
\]
The expected excess payoff $\mu - p$ is proportional to covariance with aggregate risk,
scaled by the price of risk $1/T$.

Intuition:
\begin{itemize}[leftmargin=18pt]
\item Larger $T$ (more risk tolerance in the economy) raises prices and lowers premia.
\item Greater aggregate risk exposure $\bar{\theta}$ lowers prices and raises premia.
\item Only systematic risk --- the part aligned with aggregate exposure --- is priced.
\end{itemize}

\section*{Worked Numerical Example}
Suppose a single risky payoff has
\[
\mu = 1.08, \quad \sigma^2 = 0.04, \quad p = 1.02, \quad \alpha = 2.
\]
Then
\[
\theta^* = \frac{\mu - p}{\alpha\sigma^2} = \frac{0.06}{2 \cdot 0.04} = 0.75.
\]
If aggregate risk tolerance doubles, the risk premium halves.

\section*{Exam Checklist}
Before you walk into the exam, make sure you can do the following without notes:
\begin{enumerate}[leftmargin=18pt]
\item Derive the certainty equivalent for CARA utility with normal payoffs.
\item Compute the demand vector and explain each term.
\item Aggregate demands and solve for equilibrium prices.
\item Interpret the risk premium as a covariance with aggregate exposure.
\end{enumerate}

\section*{Instructor Note}
I am an independent researcher and PhD (Georgia Institute of Technology), not currently affiliated
with any institution. These notes are my own work.

\end{document}
