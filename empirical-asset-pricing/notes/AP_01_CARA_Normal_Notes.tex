%!TEX program = xelatex
\documentclass[11pt]{article}

\usepackage[margin=1in]{geometry}
\usepackage{amsmath, amssymb, amsthm}
\usepackage{mathtools}
\usepackage{booktabs}
\usepackage{enumitem}
\usepackage{hyperref}

\setlength{\parskip}{6pt}
\setlength{\parindent}{0pt}

\newcommand{\R}{\mathbb{R}}
\newcommand{\E}{\mathbb{E}}
\newcommand{\Var}{\mathrm{Var}}
\newcommand{\Cov}{\mathrm{Cov}}

\newtheorem{definition}{Definition}
\newtheorem{lemma}{Lemma}
\newtheorem{theorem}{Theorem}

\title{AP-01 Notes: CARA-Normal Equilibrium}
\author{Dr. Ian Helfrich}
\date{\today}

\begin{document}
\maketitle
\tableofcontents
\newpage

\section{Purpose and Style}
These notes are the ``book version'' of AP-01. The slides are the live, exam-style walkthrough.
Here I slow down and write every step I expect you to reproduce under pressure. I use gross
returns and payoffs unless I say otherwise.

\section{Roadmap}
We will:
\begin{enumerate}[leftmargin=18pt]
\item Set up a one-period CARA-normal economy.
\item Derive the certainty equivalent and show the mean-variance reduction.
\item Solve the individual demand problem with full matrix calculus.
\item Aggregate demands and clear the market.
\item Interpret the equilibrium price and the risk premium.
\item Work examples and finish with qualifier-style problems and solutions.
\end{enumerate}

\section{Setup}
We study a one-period economy with $N$ risky payoffs collected in the vector $X$.
Assume
\[
X \sim \mathcal{N}(\mu, \Sigma).
\]
Agents have CARA utility
\[
U(c) = -\exp(-\alpha c), \quad \alpha > 0.
\]
Let $p \in \R^N$ be the price vector of risky payoffs. A portfolio is a vector of holdings
$\theta \in \R^N$. Final wealth is
\[
c = w_0 - p'\theta + \theta'X.
\]

\section{CARA + Normal Implies a Certainty Equivalent}
\begin{lemma}[Normal moment]
If $Y \sim \mathcal{N}(m, s^2)$, then
\[
\E[\exp(tY)] = \exp\left(tm + \frac{1}{2}t^2 s^2\right).
\]
\end{lemma}
\begin{proof}
This is the standard moment generating function of a normal random variable.
\end{proof}

\begin{lemma}[CARA-normal certainty equivalent]
If $c$ is normal, then
\[
\E[U(c)] = -\exp\left(-\alpha\E[c] + \frac{\alpha^2}{2}\Var(c)\right),
\]
so maximizing expected utility is equivalent to maximizing
\[
CE = \E[c] - \frac{\alpha}{2}\Var(c).
\]
\end{lemma}
\begin{proof}
Apply the normal moment formula with $t=-\alpha$.
\end{proof}

\section{Completing the Square (Explicit Derivation)}
Write
\[
c = m + sZ, \quad Z \sim \mathcal{N}(0,1), \quad m = \E[c], \quad s^2 = \Var(c).
\]
Then
\begin{align*}
\E[U(c)]
&= -\E\left[\exp\bigl(-\alpha(m + sZ)\bigr)\right] \\
&= -\exp(-\alpha m)\,\E\left[\exp(-\alpha s Z)\right] \\
&= -\exp(-\alpha m)\exp\left(\frac{\alpha^2 s^2}{2}\right).
\end{align*}
Taking logs (and ignoring constants) yields the certainty equivalent
\[
CE = m - \frac{\alpha}{2}s^2 = \E[c] - \frac{\alpha}{2}\Var(c).
\]

\section{Individual Demand}
Compute mean and variance:
\begin{align*}
\E[c] &= w_0 - p'\theta + \theta'\mu, \\
\Var(c) &= \theta'\Sigma\theta.
\end{align*}
So the certainty equivalent is
\[
CE = w_0 - p'\theta + \theta'\mu - \frac{\alpha}{2}\theta'\Sigma\theta.
\]

\begin{lemma}[Quadratic gradient]
If $f(\theta) = \tfrac{1}{2}\theta' A \theta$ with symmetric $A$, then
$\nabla_\theta f(\theta) = A\theta$.
\end{lemma}

Differentiate and set to zero:
\[
- p + \mu - \alpha\Sigma\theta = 0.
\]
Solve for optimal demand:
\[
\theta^* = \frac{1}{\alpha}\Sigma^{-1}(\mu - p).
\]

\textbf{Interpretation.} Demand is increasing in expected payoff $\mu$, decreasing in price $p$,
and scaled down by risk aversion and covariance risk.

\section{Aggregate Risk Tolerance}
Let agent $h$ have risk aversion $\alpha_h$. Summing demands yields
\[
\sum_h \theta_h^* = \left(\sum_h \frac{1}{\alpha_h}\right)\Sigma^{-1}(\mu - p).
\]
\begin{definition}[Total risk tolerance]
Define
\[
T = \sum_h \frac{1}{\alpha_h}.
\]
\end{definition}
Then aggregate demand is
\[
\sum_h \theta_h^* = T\Sigma^{-1}(\mu - p).
\]

\section{Market Clearing and Equilibrium Price}
Let $\bar{\theta}$ be the fixed supply of risky payoffs. Market clearing requires
\[
\bar{\theta} = T\Sigma^{-1}(\mu - p).
\]
Solve for the price vector:
\[
 p = \mu - \frac{1}{T}\Sigma\bar{\theta}.
\]

\begin{theorem}[CARA-normal equilibrium price]
In a CARA-normal economy with aggregate supply $\bar{\theta}$, equilibrium prices satisfy
\[
 p = \mu - \frac{1}{T}\Sigma\bar{\theta}.
\]
\end{theorem}
\begin{proof}
Combine individual demand with market clearing as shown above.
\end{proof}

\section{Risk Premia and Interpretation}
Rewrite the equilibrium condition as
\[
\mu - p = \frac{1}{T}\Sigma\bar{\theta}.
\]
Thus expected excess payoffs are proportional to covariance with aggregate risk.
The scalar $1/T$ is the price of risk.

\textbf{Economic intuition:}
\begin{itemize}[leftmargin=18pt]
\item Larger $T$ (more risk tolerance) raises prices and lowers premia.
\item Higher aggregate risk exposure $\bar{\theta}$ lowers prices.
\item Only systematic risk matters; idiosyncratic risk washes out in aggregation.
\end{itemize}

\section{Pricing Kernel View (Optional)}
With exponential utility, the (unnormalized) pricing kernel is
\[
m(\omega) \propto \exp\bigl(-\alpha c(\omega)\bigr).
\]
In a CARA-normal setting, the price formula collapses to a linear covariance adjustment.
This is the bridge to mean-variance pricing and, later, the CAPM.

\section{Vector Completion of the Square (Full Algebra)}
Write wealth as $c = w_0 - p'\theta + \theta'X$. Then
\[
\E[c] = w_0 - p'\theta + \theta'\mu,
\quad
\Var(c) = \theta'\Sigma\theta.
\]
The certainty equivalent becomes
\[
CE = w_0 - p'\theta + \theta'\mu - \frac{\alpha}{2}\theta'\Sigma\theta.
\]
This is a strictly concave quadratic in $\theta$. The unique maximizer satisfies
\[
\alpha\Sigma\theta = \mu - p
\quad\Rightarrow\quad
\theta^* = \frac{1}{\alpha}\Sigma^{-1}(\mu - p).
\]

\section{Representative Agent Interpretation}
If we aggregate to a representative agent with total risk tolerance $T$, the pricing kernel
is proportional to $\exp(-\frac{1}{T}\bar{\theta}'X)$. This reproduces the same pricing
equation:
\[
p = \E[X] - \frac{1}{T}\Cov(X, \bar{\theta}'X).
\]
This is the exact same object written as a covariance adjustment.

\section{Worked Examples}
\subsection*{Example 1: Single Asset}
Suppose a single risky payoff has
\[
\mu = 1.08, \quad \sigma^2 = 0.04, \quad p = 1.02, \quad \alpha = 2.
\]
Then
\[
\theta^* = \frac{\mu - p}{\alpha\sigma^2} = \frac{0.06}{2 \cdot 0.04} = 0.75.
\]

\subsection*{Example 2: Two Assets, One Unit Supply}
Let
\[
\mu = \begin{bmatrix}1.08 \\ 1.12\end{bmatrix}, \quad
\Sigma = \begin{bmatrix}0.04 & 0.01 \\ 0.01 & 0.09\end{bmatrix}, \quad
\bar{\theta} = \begin{bmatrix}1 \\ 1\end{bmatrix}, \quad T = 2.
\]
Compute
\[
\Sigma\bar{\theta} = \begin{bmatrix}0.05 \\ 0.10\end{bmatrix},
\quad
 p = \mu - \frac{1}{2}\begin{bmatrix}0.05 \\ 0.10\end{bmatrix}
= \begin{bmatrix}1.055 \\ 1.07\end{bmatrix}.
\]

\subsection*{Example 3: Heterogeneous Means}
Suppose two agents share $\Sigma$ but have different beliefs $\mu_1$ and $\mu_2$.
Then
\[
\sum_h \theta_h^* = \Sigma^{-1}\left(\sum_h \frac{\mu_h}{\alpha_h} - pT\right).
\]
Thus equilibrium prices depend on the risk-tolerance-weighted average of beliefs.

\subsection*{Example 4: Risky Supply Tilt}
Let
\[
\Sigma = \begin{bmatrix}0.09 & 0.02 \\\\ 0.02 & 0.04\end{bmatrix},
\quad
\bar{\theta} = \begin{bmatrix}2 \\\\ 0.5\end{bmatrix},
\quad T = 2.
\]
Then
\[
\Sigma\bar{\theta} = \begin{bmatrix}0.20 \\\\ 0.08\end{bmatrix},
\quad
p = \mu - \frac{1}{2}\begin{bmatrix}0.20 \\\\ 0.08\end{bmatrix}.
\]
The high-supply asset receives a larger price discount.

\section{Comparative Statics}
\begin{itemize}[leftmargin=18pt]
\item If $T$ doubles, risk premia halve.
\item If $\Sigma$ increases (more risk), prices fall.
\item If supply $\bar{\theta}$ rises, prices fall proportionally to risk exposure.
\end{itemize}

\section{Qualifier-Style Problem A (Full Walkthrough)}
\textbf{Problem.} There are $N$ risky payoffs with $X \sim \mathcal{N}(\mu, \Sigma)$.
Agents have CARA utility $U(c) = -\exp(-\alpha c)$. Supply is $\bar{\theta}$.
Derive equilibrium prices and interpret the risk premium.

\subsection*{Solution}
Step 1. Write wealth as $c = w_0 - p'\theta + \theta'X$.

Step 2. Use CARA-normal to get
\[
CE = \E[c] - \frac{\alpha}{2}\Var(c)
= w_0 - p'\theta + \theta'\mu - \frac{\alpha}{2}\theta'\Sigma\theta.
\]

Step 3. Differentiate and solve:
\[
-p + \mu - \alpha\Sigma\theta = 0
\quad\Rightarrow\quad
\theta^* = \frac{1}{\alpha}\Sigma^{-1}(\mu - p).
\]

Step 4. Aggregate demands: $\sum_h \theta_h^* = T\Sigma^{-1}(\mu - p)$.

Step 5. Market clearing $\sum_h \theta_h^* = \bar{\theta}$ yields
\[
p = \mu - \frac{1}{T}\Sigma\bar{\theta}.
\]

Step 6. Risk premium:
\[
\mu - p = \frac{1}{T}\Sigma\bar{\theta}.
\]
Interpretation: only covariance with aggregate exposure is priced, scaled by $1/T$.

\section{Qualifier-Style Problem B (Heterogeneous Beliefs)}
\textbf{Problem.} Two agents share the same covariance $\Sigma$ but have different
means $\mu_1, \mu_2$ and risk aversions $\alpha_1, \alpha_2$.
Find the equilibrium price.

\subsection*{Solution}
Each agent demands
\[
\theta_h^* = \frac{1}{\alpha_h}\Sigma^{-1}(\mu_h - p).
\]
Sum:
\[
\sum_h \theta_h^*
= \Sigma^{-1}\left(\sum_h \frac{\mu_h}{\alpha_h} - p\sum_h \frac{1}{\alpha_h}\right).
\]
Define $T = \sum_h 1/\alpha_h$ and $\bar{\mu} = \frac{1}{T}\sum_h \mu_h/\alpha_h$.
Then market clearing gives
\[
p = \bar{\mu} - \frac{1}{T}\Sigma\bar{\theta}.
\]
Thus prices are driven by the risk-tolerance-weighted average belief.

\section{Matrix Identities (Quick Reference)}
These appear repeatedly in qualifiers:
\begin{itemize}[leftmargin=18pt]
\item If $f(\theta) = a'\theta$, then $\nabla_\theta f = a$.
\item If $f(\theta) = \tfrac{1}{2}\theta' A \theta$ with symmetric $A$, then
      $\nabla_\theta f = A\theta$.
\item If $A$ is invertible, then $A\theta = b \Rightarrow \theta = A^{-1}b$.
\end{itemize}

\section{Step-by-Step Blueprint (Exam Edition)}
When time is tight, follow this exact structure:
\begin{enumerate}[leftmargin=18pt]
\item Write $c = w_0 - p'\theta + \theta'X$ and state $X \sim \mathcal{N}(\mu,\Sigma)$.
\item Replace utility with $CE = \E[c] - \tfrac{\alpha}{2}\Var(c)$.
\item Solve $-p + \mu - \alpha\Sigma\theta = 0$ for $\theta^*$.
\item Aggregate and define $T = \sum_h 1/\alpha_h$.
\item Clear the market and report $p = \mu - \Sigma\bar{\theta}/T$.
\item Interpret the risk premium as covariance with aggregate exposure.
\end{enumerate}

\section{Common Pitfalls}
\begin{itemize}[leftmargin=18pt]
\item Forgetting that $p$ is a vector of prices, not returns.
\item Dropping the matrix inverse in $\theta^* = \frac{1}{\alpha}\Sigma^{-1}(\mu - p)$.
\item Confusing $T$ (risk tolerance) with average risk aversion.
\item Skipping the aggregation step and writing the price formula by memory.
\end{itemize}

\section{Exam Checklist}
Before exam day, you should be able to do the following without notes:
\begin{enumerate}[leftmargin=18pt]
\item Derive the certainty equivalent for CARA utility under normality.
\item Compute the demand vector and interpret each term.
\item Aggregate demands and solve for equilibrium prices.
\item Translate prices into expected excess payoffs and interpret risk premia.
\end{enumerate}

\section{Practice Problems}
\begin{enumerate}[leftmargin=18pt]
\item Single asset: derive $\theta^*$ and solve for $p$ given $\bar{\theta} = 1$.
\item Two agents: compute $T$ with $\alpha_1=1$ and $\alpha_2=2$ and solve for $p$.
\item Comparative statics: show how $p$ changes when $\Sigma$ doubles.
\item Prove the certainty equivalent formula directly by completing the square.
\item Explain how the equilibrium price changes if agents face different $\mu$ but the same $\Sigma$.
\item Suppose $\Sigma$ is diagonal. Interpret prices asset by asset.
\item Derive the price formula using a representative agent and the pricing kernel.
\item Show that if $\bar{\theta}=0$ then $p=\mu$ and explain the economic meaning.
\end{enumerate}

\section{Solutions}
\subsection*{Solution 1}
With one asset, the FOC gives $\theta^* = (\mu - p)/(\alpha\sigma^2)$.
Market clearing $\theta^* = 1$ implies $p = \mu - \alpha\sigma^2$.

\subsection*{Solution 2}
Risk tolerance is $T = 1/1 + 1/2 = 1.5$.
The equilibrium price is $p = \mu - \Sigma\bar{\theta}/T$.

\subsection*{Solution 3}
If $\Sigma$ doubles, then $\Sigma\bar{\theta}$ doubles, so $p$ falls by
$\frac{1}{T}(\Sigma\bar{\theta})$.

\subsection*{Solution 4}
Write $c = m + sZ$ with $Z \sim \mathcal{N}(0,1)$. Then
$\E[\exp(-\alpha c)] = \exp(-\alpha m + \tfrac{1}{2}\alpha^2 s^2)$.
Thus the certainty equivalent is $m - \tfrac{\alpha}{2}s^2$.

\subsection*{Solution 5}
With heterogeneous $\mu_h$, each agent has demand
$\theta_h^* = \frac{1}{\alpha_h}\Sigma^{-1}(\mu_h - p)$.
Summing gives $\sum_h \theta_h^* = \Sigma^{-1}(\sum_h (\mu_h/\alpha_h) - pT)$.
Thus the price depends on the risk-tolerance-weighted average of means.

\subsection*{Solution 6}
If $\Sigma$ is diagonal, then each asset is priced independently:
$p_i = \mu_i - (\sigma_i^2 \bar{\theta}_i)/T$.

\subsection*{Solution 7}
With representative agent risk tolerance $T$, the pricing kernel is proportional to
$\exp(-\frac{1}{T}\bar{\theta}'X)$. Then
$p = \E[X] - \frac{1}{T}\Cov(X, \bar{\theta}'X)$, which matches the equilibrium price.

\subsection*{Solution 8}
If $\bar{\theta}=0$, the economy has no aggregate risk exposure. Then
$p = \mu$. Prices equal expected payoffs because there is no risk to price.

\section*{Instructor Note}
I am an independent researcher and PhD (Georgia Institute of Technology), not currently affiliated
with any institution. These notes are my own work.

\end{document}
