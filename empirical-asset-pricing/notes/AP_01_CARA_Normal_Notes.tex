%!TEX program = xelatex
\documentclass[11pt]{article}

\usepackage[margin=1in]{geometry}
\usepackage{amsmath, amssymb, amsthm}
\usepackage{mathtools}
\usepackage{booktabs}
\usepackage{enumitem}
\usepackage{hyperref}

\setlength{\parskip}{6pt}
\setlength{\parindent}{0pt}

\newcommand{\R}{\mathbb{R}}
\newcommand{\E}{\mathbb{E}}
\newcommand{\Var}{\mathrm{Var}}
\newcommand{\Cov}{\mathrm{Cov}}

\title{AP-01 Notes: CARA-Normal Equilibrium}
\author{Dr. Ian Helfrich}
\date{\today}

\begin{document}
\maketitle

\section*{Purpose of These Notes}
These notes are the ``book version'' of AP-01. The slides are the live, exam-style walkthrough.
Here I slow down and write every step I expect you to reproduce under time pressure.

\section*{Roadmap}
We will:
\begin{enumerate}[leftmargin=18pt]
\item Set up a one-period CARA-normal economy.
\item Derive the certainty equivalent and show the mean-variance form.
\item Solve the individual demand problem.
\item Aggregate demands and clear the market.
\item Interpret the equilibrium price and the risk premium.
\end{enumerate}

\section*{1. Setup}
We study a one-period economy with $N$ risky payoffs collected in the vector $X$.
Assume
\[
X \sim \mathcal{N}(\mu, \Sigma).
\]
Agents have CARA utility
\[
U(c) = -\exp(-\alpha c), \quad \alpha > 0.
\]
Let $p \in \R^N$ be the price vector of risky payoffs. A portfolio is a vector of holdings
$\theta \in \R^N$.

Final wealth is
\[
c = w_0 - p'\theta + \theta'X.
\]

\section*{2. CARA + Normal Implies a Certainty Equivalent}
Key fact: if $Y \sim \mathcal{N}(m, s^2)$, then
\[
\E[\exp(-\alpha Y)] = \exp\left(-\alpha m + \frac{\alpha^2}{2}s^2\right).
\]
Proof sketch: the moment generating function of a normal variable gives
\(\E[\exp(tY)] = \exp(tm + \tfrac{1}{2}t^2 s^2)\). Set $t=-\alpha$.

Therefore, if $c$ is normal,
\[
\E[U(c)] = -\exp\left(-\alpha\E[c] + \frac{\alpha^2}{2}\Var(c)\right).
\]
Maximizing expected utility is equivalent to maximizing the certainty equivalent
\[
CE = \E[c] - \frac{\alpha}{2}\Var(c).
\]
This is why CARA-normal problems become mean-variance problems.

\section*{3. Individual Demand}
Compute mean and variance:
\begin{align*}
\E[c] &= w_0 - p'\theta + \theta'\mu, \\
\Var(c) &= \theta'\Sigma\theta.
\end{align*}
So
\[
CE = w_0 - p'\theta + \theta'\mu - \frac{\alpha}{2}\theta'\Sigma\theta.
\]
Differentiate with respect to $\theta$ and set to zero:
\[
- p + \mu - \alpha\Sigma\theta = 0.
\]
Solve for optimal demand:
\[
\theta^* = \frac{1}{\alpha}\Sigma^{-1}(\mu - p).
\]

Interpretation: demand is increasing in expected payoff $\mu$, decreasing in price $p$,
and scaled down by risk aversion and covariance risk.

\section*{4. Aggregate Risk Tolerance}
Let agent $h$ have risk aversion $\alpha_h$. Summing demands yields
\[
\sum_h \theta_h^* = \left(\sum_h \frac{1}{\alpha_h}\right)\Sigma^{-1}(\mu - p).
\]
Define total risk tolerance
\[
T = \sum_h \frac{1}{\alpha_h}.
\]
Then aggregate demand is
\[
\sum_h \theta_h^* = T\Sigma^{-1}(\mu - p).
\]

\section*{5. Market Clearing and Equilibrium Price}
Let $\bar{\theta}$ be the fixed supply of risky payoffs. Market clearing implies
\[
\bar{\theta} = T\Sigma^{-1}(\mu - p).
\]
Solve for price:
\[
 p = \mu - \frac{1}{T}\Sigma\bar{\theta}.
\]
This is the equilibrium pricing equation.

\section*{6. Risk Premia and Interpretation}
Rewrite as
\[
\mu - p = \frac{1}{T}\Sigma\bar{\theta}.
\]
Thus expected excess payoffs are proportional to covariance with aggregate risk.
The scalar $1/T$ is the price of risk.

Economic intuition:
\begin{itemize}[leftmargin=18pt]
\item Larger $T$ (more risk tolerance) raises prices and lowers premia.
\item Higher aggregate risk exposure $\bar{\theta}$ lowers prices.
\item Only systematic risk matters; idiosyncratic risk washes out in aggregation.
\end{itemize}

\section*{7. Worked Example (Two Assets)}
Let
\[
\mu = \begin{bmatrix}1.08 \\ 1.12\end{bmatrix}, \quad
\Sigma = \begin{bmatrix}0.04 & 0.01 \\ 0.01 & 0.09\end{bmatrix}, \quad
\bar{\theta} = \begin{bmatrix}1 \\ 1\end{bmatrix}, \quad T = 2.
\]
Compute
\[
\Sigma\bar{\theta} = \begin{bmatrix}0.05 \\ 0.10\end{bmatrix}.
\]
Then
\[
 p = \mu - \frac{1}{2}\begin{bmatrix}0.05 \\ 0.10\end{bmatrix}
= \begin{bmatrix}1.055 \\ 1.07\end{bmatrix}.
\]

\section*{8. Comparative Statics}
\begin{itemize}[leftmargin=18pt]
\item If $T$ doubles, risk premia halve.
\item If $\Sigma$ increases (more risk), prices fall.
\item If supply $\bar{\theta}$ rises, prices fall proportionally to risk exposure.
\end{itemize}

\section*{9. Exam Checklist}
Before exam day, you should be able to do the following without notes:
\begin{enumerate}[leftmargin=18pt]
\item Derive the certainty equivalent for CARA utility under normality.
\item Compute the demand vector and interpret each term.
\item Aggregate demands and solve for equilibrium prices.
\item Translate prices into expected excess payoffs and interpret risk premia.
\end{enumerate}

\section*{10. Practice Problems}
\begin{enumerate}[leftmargin=18pt]
\item Single asset: derive $\theta^*$ and solve for $p$ given $\bar{\theta} = 1$.
\item Two agents: compute $T$ with $\alpha_1=1$ and $\alpha_2=2$ and solve for $p$.
\item Comparative statics: show how $p$ changes when $\Sigma$ doubles.
\end{enumerate}

\section*{Instructor Note}
I am an independent researcher and PhD (Georgia Institute of Technology), not currently affiliated
with any institution. These notes are my own work.

\end{document}
