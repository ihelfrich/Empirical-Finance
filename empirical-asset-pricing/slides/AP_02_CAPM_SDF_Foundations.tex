%!TEX program = xelatex
\documentclass[10pt]{beamer}

% Theme
\usetheme[progressbar=frametitle]{metropolis}
\usepackage{appendixnumberbeamer}

% Color palette (RA-inspired, finance-leaning)
\definecolor{StudyBlue}{HTML}{2E86C1}
\definecolor{StudyTeal}{HTML}{17A2B8}
\definecolor{StudyGrey}{HTML}{495057}
\definecolor{StudyLight}{HTML}{F8F9FA}
\definecolor{StudyGold}{HTML}{B8860B}

\setbeamercolor{primary}{bg=StudyBlue}
\setbeamercolor{secondary}{bg=StudyGrey}
\setbeamercolor{accent}{bg=StudyTeal}
\setbeamercolor{frametitle}{bg=StudyBlue, fg=white}
\setbeamercolor{progress bar}{fg=StudyTeal, bg=StudyGrey}

% Font sizing
\setbeamerfont{normal text}{size=\footnotesize}
\setbeamerfont{structure}{size=\footnotesize}
\setbeamerfont{frametitle}{size=\large}
\setbeamerfont{framesubtitle}{size=\scriptsize}

% Packages
\usepackage{amsmath}
\usepackage{amssymb}
\usepackage{amsthm}
\usepackage{mathtools}
\usepackage{graphicx}
\usepackage{tikz}
\usepackage{array}
\usepackage{booktabs}
\usepackage{tcolorbox}
\usepackage{bookmark}

% tcolorbox setup
\tcbuselibrary{theorems,skins}

% Spacing
\setlength{\parskip}{2pt}
\setlength{\itemsep}{1pt}

% Bullet styles
\setbeamercolor{item}{fg=StudyBlue}
\setbeamercolor{subitem}{fg=StudyBlue}
\setbeamercolor{subsubitem}{fg=StudyBlue}
\setbeamertemplate{itemize item}{\normalsize$\bullet$}
\setbeamertemplate{itemize subitem}{\large$\circ$}
\setbeamertemplate{itemize subsubitem}{\Large$\diamond$}

\setbeamercolor{enumerate item}{fg=StudyBlue}
\setbeamercolor{enumerate subitem}{fg=StudyBlue}
\setbeamercolor{enumerate subsubitem}{fg=StudyBlue}
\setbeamertemplate{enumerate item}{\textbf{\arabic{enumi}.}}
\setbeamertemplate{enumerate subitem}{\textbf{\alph{enumii})}}
\setbeamertemplate{enumerate subsubitem}{\textbf{\roman{enumiii}.}}

% Math shortcuts
\newcommand{\R}{\mathbb{R}}
\newcommand{\E}{\mathbb{E}}
\newcommand{\Var}{\mathrm{Var}}
\newcommand{\Cov}{\mathrm{Cov}}

% Box styles
\newtcolorbox{roundedbox}{
    colback=StudyBlue!5,
    colframe=StudyBlue!80,
    rounded corners,
    fonttitle=\bfseries
}

\newtcolorbox{definitionbox}[1][]{
    colback=StudyBlue!5,
    colframe=StudyBlue!80,
    fonttitle=\bfseries,
    title=Definition,
    before upper={
        \setbeamercolor{itemize item}{fg=StudyBlue}
        \setbeamercolor{itemize subitem}{fg=StudyBlue}
        \setbeamercolor{itemize subsubitem}{fg=StudyBlue}
    },
    #1
}

\newtcolorbox{theorembox}[1][]{
    colback=StudyTeal!5,
    colframe=StudyTeal!80,
    fonttitle=\bfseries,
    title=Theorem,
    before upper={
        \setbeamercolor{itemize item}{fg=StudyTeal}
        \setbeamercolor{itemize subitem}{fg=StudyTeal}
        \setbeamercolor{itemize subsubitem}{fg=StudyTeal}
    },
    #1
}

\newtcolorbox{examplebox}[1][]{
    colback=StudyGrey!5,
    colframe=StudyGrey!80,
    fonttitle=\bfseries,
    title=Example,
    before upper={
        \setbeamercolor{itemize item}{fg=StudyGrey}
        \setbeamercolor{itemize subitem}{fg=StudyGrey}
        \setbeamercolor{itemize subsubitem}{fg=StudyGrey}
    },
    #1
}

\newtcolorbox{notebox}[1][]{
    colback=black!5,
    colframe=black!90,
    fonttitle=\bfseries,
    fontupper=\footnotesize,
    title=Intuition,
    before upper={
        \setbeamercolor{itemize item}{fg=black}
        \setbeamercolor{itemize subitem}{fg=black}
        \setbeamercolor{itemize subsubitem}{fg=black}
    },
    #1
}

\newtcolorbox{proofbox}[1][]{
    colback=StudyLight,
    colframe=StudyGrey!50,
    fonttitle=\bfseries,
    title=Proof,
    before upper={
        \setbeamercolor{itemize item}{fg=StudyGrey}
        \setbeamercolor{itemize subitem}{fg=StudyGrey}
        \setbeamercolor{itemize subsubitem}{fg=StudyGrey}
    },
    #1
}

\newtcolorbox{keypoint}[1][]{
    colback=yellow!10,
    colframe=orange!80,
    fonttitle=\bfseries,
    title=Key Point,
    before upper={
        \setbeamercolor{itemize item}{fg=orange}
        \setbeamercolor{itemize subitem}{fg=orange}
        \setbeamercolor{itemize subsubitem}{fg=orange}
    },
    #1
}

\newtcolorbox{problembox}[1][]{
    colback=StudyGold!10,
    colframe=StudyGold!90,
    fonttitle=\bfseries,
    title=Problem,
    before upper={
        \setbeamercolor{itemize item}{fg=StudyGold}
        \setbeamercolor{itemize subitem}{fg=StudyGold}
        \setbeamercolor{itemize subsubitem}{fg=StudyGold}
    },
    #1
}

\title{Empirical Asset Pricing: Qualifier Prep}
\subtitle{AP-02: CAPM and SDF Foundations}
\author{Dr. Ian Helfrich}
\date{\today}

\begin{document}

\maketitle

\begin{frame}{From the Instructor}
\small
I write these slides as a working researcher and teacher. The goal is to make the CAPM feel like
an unavoidable result rather than a memorized formula. We will move slowly, show all algebra,
and repeatedly connect each line to the economics it represents.

\medskip
I am an independent researcher and PhD (Georgia Institute of Technology), not currently affiliated
with an institution. These notes reflect my own voice and method.
\end{frame}

\begin{frame}{Roadmap}
\small
\begin{enumerate}
\item \textbf{Build the Mean-Variance Problem}
\begin{itemize}
\item Portfolio choice with a risk-free asset
\item Efficient frontier and tangency portfolio
\end{itemize}
\item \textbf{Derive the CAPM Step-by-Step}
\begin{itemize}
\item Market clearing and the market portfolio
\item Beta pricing and risk premia
\end{itemize}
\item \textbf{Translate to the SDF View}
\begin{itemize}
\item SDF definition and pricing equation
\item Linear SDF implies CAPM
\end{itemize}
\item \textbf{Qualifier Problems + Variants}
\begin{itemize}
\item Two full exam-style problems
\item Common pitfalls and extensions
\end{itemize}
\end{enumerate}
\end{frame}

\section{Motivation}

\begin{frame}{Why CAPM Still Matters}
\small
\begin{itemize}
\item The CAPM is not an empirical truth; it is a disciplined equilibrium benchmark.
\item It teaches you how to translate preferences and technology into pricing restrictions.
\item Most qualifier questions use CAPM logic as a baseline to test deeper understanding.
\end{itemize}

\begin{notebox}
The qualifier test is often: can you recreate the logic under time pressure,
not can you recite the final formula.
\end{notebox}
\end{frame}

\begin{frame}{A Preview of the Destination}
\small
We will show that in equilibrium with a risk-free asset,
\[
\E[R_i] - R_f = \beta_i\bigl(\E[R_M] - R_f\bigr),
\qquad
\beta_i = \frac{\Cov(R_i, R_M)}{\Var(R_M)}.
\]

\medskip
Then we will prove the same equation from the SDF condition
\[
\E[m R_i] = 1,
\qquad
m = a - b R_M.
\]

\begin{keypoint}
Two derivations, one idea: only covariance with the market is priced.
\end{keypoint}
\end{frame}

\section{Mean-Variance Foundation}

\begin{frame}{Setup: Assets and Portfolios}
\small
\begin{itemize}
\item $N$ risky assets with returns $R \in \R^N$, mean $\mu$, covariance $\Sigma$.
\item A risk-free asset with return $R_f$.
\item A portfolio is weights $w \in \R^N$ in risky assets; remainder in the risk-free asset.
\end{itemize}

\begin{definitionbox}
The portfolio return is
\[
R_p = R_f + w' (R - R_f \mathbf{1}).
\]
\end{definitionbox}
\end{frame}

\begin{frame}{Mean and Variance of the Portfolio}
\small
\begin{align*}
\E[R_p] &= R_f + w'(\mu - R_f \mathbf{1}), \\
\Var(R_p) &= w' \Sigma w.
\end{align*}

\begin{notebox}
The mean depends only on the projection of $w$ on excess means,
while the variance depends on $\Sigma$.
\end{notebox}
\end{frame}

\begin{frame}{Mean-Variance Optimization Problem}
\small
The canonical investor solves
\[
\max_{w} \; \E[R_p] - \frac{\gamma}{2} \Var(R_p).
\]
Substitute the formulas:
\[
\max_{w} \; R_f + w'(\mu - R_f \mathbf{1}) - \frac{\gamma}{2} w'\Sigma w.
\]

\begin{keypoint}
All investors choose from the same set of efficient portfolios.
Only $\gamma$ moves them along the line.
\end{keypoint}
\end{frame}

\begin{frame}{Solve the First-Order Condition}
\small
Take the gradient with respect to $w$:
\[
\mu - R_f \mathbf{1} - \gamma \Sigma w = 0.
\]
Solve:
\[
\Sigma w = \frac{1}{\gamma}(\mu - R_f \mathbf{1}).
\]
Therefore,
\[
 w = \frac{1}{\gamma} \Sigma^{-1}(\mu - R_f \mathbf{1}).
\]

\begin{notebox}
Up to a scaling factor, everyone holds the same risky portfolio.
\end{notebox}
\end{frame}

\begin{frame}{The Tangency Portfolio}
\small
Define the tangency portfolio weights
\[
 w_T = \frac{\Sigma^{-1}(\mu - R_f \mathbf{1})}{\mathbf{1}'\Sigma^{-1}(\mu - R_f \mathbf{1})}.
\]
This portfolio maximizes the Sharpe ratio.

\begin{keypoint}
Every mean-variance investor holds the tangency portfolio and
adjusts overall risk by scaling with $R_f$.
\end{keypoint}
\end{frame}

\begin{frame}{Market Clearing Logic}
\small
In equilibrium, the aggregate risky portfolio must equal the market portfolio $M$.
If all agents hold a multiple of the tangency portfolio, then
\[
 w_M = w_T.
\]

\begin{notebox}
This is the key equilibrium step: the market portfolio is efficient.
Once you accept that, CAPM follows.
\end{notebox}
\end{frame}

\section{CAPM Derivation}

\begin{frame}{A Key Relationship: Covariance with the Market}
\small
Let $R_M$ be the market return. For any asset $i$:
\[
\Cov(R_i, R_M) = \sigma_{iM}.
\]

From mean-variance efficiency we can show
\[
\mu_i - R_f = \lambda \sigma_{iM}
\]
for some scalar $\lambda$.
\end{frame}

\begin{frame}{Derivation Step-by-Step}
\small
We know that $w_M$ satisfies
\[
\Sigma w_M = \kappa (\mu - R_f \mathbf{1})
\]
for some scalar $\kappa$.

Take the $i$th element:
\[
(\Sigma w_M)_i = \kappa (\mu_i - R_f).
\]
But $(\Sigma w_M)_i = \Cov(R_i, R_M)$, so
\[
\mu_i - R_f = \frac{1}{\kappa} \Cov(R_i, R_M).
\]
\end{frame}

\begin{frame}{Identify the Slope}
\small
Apply the same formula to the market itself:
\[
\mu_M - R_f = \frac{1}{\kappa} \Var(R_M).
\]
Solve for $1/\kappa$ and substitute:
\[
\mu_i - R_f = \frac{\Cov(R_i, R_M)}{\Var(R_M)} (\mu_M - R_f).
\]
This is the CAPM.
\end{frame}

\begin{frame}{Beta Form}
\small
Define
\[
\beta_i = \frac{\Cov(R_i, R_M)}{\Var(R_M)}.
\]
Then
\[
\E[R_i] = R_f + \beta_i (\E[R_M] - R_f).
\]

\begin{keypoint}
Only covariance with the market matters; idiosyncratic risk is diversifiable.
\end{keypoint}
\end{frame}

\begin{frame}{Geometric Interpretation}
\small
In mean-variance space, the efficient frontier with a risk-free asset becomes a line.
The line passes through $R_f$ and is tangent to the risky-asset frontier.

\begin{notebox}
The tangency point is the market portfolio. The slope of this line
is the Sharpe ratio of the market.
\end{notebox}
\end{frame}

\section{SDF Formulation}

\begin{frame}{The Stochastic Discount Factor}
\small
\begin{definitionbox}
A stochastic discount factor (SDF) $m$ is a random variable such that
\[
\E[m R_i] = 1
\]
for every traded asset $i$.
\end{definitionbox}

\begin{notebox}
This is the modern pricing equation. It is equivalent to no-arbitrage plus
market completeness in this simple setting.
\end{notebox}
\end{frame}

\begin{frame}{Linear SDF Assumption}
\small
Suppose the SDF is linear in the market return:
\[
 m = a - b R_M.
\]
Use the pricing equations for the risk-free asset and the market:
\begin{align*}
\E[m] R_f &= 1, \\
\E[m R_M] &= 1.
\end{align*}
We can solve for $a$ and $b$ explicitly.
\end{frame}

\begin{frame}{Solve for $a$ and $b$}
\small
First equation:
\[
R_f \E[a - b R_M] = 1 \quad \Rightarrow \quad a - b \E[R_M] = \frac{1}{R_f}.
\]
Second equation:
\[
\E[(a - b R_M) R_M] = 1 \quad \Rightarrow \quad a \E[R_M] - b \E[R_M^2] = 1.
\]

Solve this linear system for $a$ and $b$.
\end{frame}

\begin{frame}{Key Covariance Identity}
\small
A cleaner way is to use covariance:
\begin{align*}
\E[m R_i] &= \E[m] \E[R_i] + \Cov(m, R_i) = 1.
\end{align*}
Rearrange:
\[
\E[R_i] - R_f = -\frac{\Cov(m, R_i)}{\E[m]}.
\]

If $m = a - b R_M$, then $\Cov(m, R_i) = -b \Cov(R_M, R_i)$.
\end{frame}

\begin{frame}{Recover CAPM from the SDF}
\small
Substitute the covariance expression:
\[
\E[R_i] - R_f = \frac{b}{\E[m]} \Cov(R_i, R_M).
\]
Apply the same formula to asset $M$ to identify the slope:
\[
\E[R_M] - R_f = \frac{b}{\E[m]} \Var(R_M).
\]
Divide the two equations to obtain CAPM:
\[
\E[R_i] - R_f = \beta_i (\E[R_M] - R_f).
\]
\end{frame}

\section{Worked Examples}

\begin{frame}{Example 1: Compute Beta and Expected Return}
\small
Suppose:
\begin{itemize}
\item $R_f = 1.02$ (2\% per period),
\item $\E[R_M] = 1.10$,
\item $\Var(R_M) = 0.04$,
\item $\Cov(R_i, R_M) = 0.06$.
\end{itemize}
Then
\[
\beta_i = \frac{0.06}{0.04} = 1.5.
\]
So
\[
\E[R_i] = 1.02 + 1.5(1.10 - 1.02) = 1.02 + 1.5(0.08) = 1.14.
\]
\end{frame}

\begin{frame}{Example 2: From Covariance Matrix to Betas}
\small
Let
\[
\Sigma = \begin{bmatrix}
0.04 & 0.01 \\
0.01 & 0.09
\end{bmatrix},
\qquad
w_M = \begin{bmatrix}0.6 \\ 0.4\end{bmatrix}.
\]
Compute
\[
\Var(R_M) = w_M' \Sigma w_M = 0.6^2(0.04) + 2(0.6)(0.4)(0.01) + 0.4^2(0.09) = 0.0432.
\]
For asset 1,
\[
\Cov(R_1, R_M) = e_1'\Sigma w_M = 0.6(0.04) + 0.4(0.01) = 0.028.
\]
Hence $\beta_1 = 0.028/0.0432 \approx 0.648$.
\end{frame}

\begin{frame}{Example 3: Tangency Portfolio by Hand}
\small
Suppose
\[
\mu = \begin{bmatrix}1.08 \\ 1.12\end{bmatrix},
\quad
R_f = 1.02,
\quad
\Sigma = \begin{bmatrix}0.04 & 0.01 \\ 0.01 & 0.09\end{bmatrix}.
\]
Compute
\[
\mu - R_f \mathbf{1} = \begin{bmatrix}0.06 \\ 0.10\end{bmatrix}.
\]
Then
\[
\Sigma^{-1}(\mu - R_f \mathbf{1}) = \frac{1}{0.0035}
\begin{bmatrix}0.09 & -0.01 \\ -0.01 & 0.04\end{bmatrix}
\begin{bmatrix}0.06 \\ 0.10\end{bmatrix}
= \frac{1}{0.0035}
\begin{bmatrix}0.0044 \\ 0.0034\end{bmatrix}.
\]
Normalize to sum to 1: $w_T \propto (0.0044, 0.0034)$.
\end{frame}

\section{Qualifier-Style Problems}

\begin{frame}{Problem A: Full CAPM Derivation}
\small
\begin{problembox}
An economy has $N$ risky assets with mean $\mu$ and covariance $\Sigma$ and a risk-free asset $R_f$.
Assume mean-variance preferences. Show that in equilibrium $\E[R_i] - R_f = \beta_i(\E[R_M] - R_f)$.
\end{problembox}

\textbf{Outline:}
\begin{enumerate}
\item Write the mean-variance objective and FOC.
\item Show that the tangency portfolio has weights proportional to $\Sigma^{-1}(\mu - R_f \mathbf{1})$.
\item Market clearing implies $w_M = w_T$.
\item Convert $\Sigma w_M = \kappa (\mu - R_f \mathbf{1})$ to a covariance statement.
\item Solve for $\kappa$ using the market itself.
\end{enumerate}
\end{frame}

\begin{frame}{Problem A: Step-by-Step Solution (1/2)}
\small
Start with the investor's problem:
\[
\max_w R_f + w'(\mu - R_f \mathbf{1}) - \frac{\gamma}{2} w'\Sigma w.
\]
FOC:
\[
\mu - R_f \mathbf{1} - \gamma \Sigma w = 0.
\]
Thus
\[
 w = \frac{1}{\gamma} \Sigma^{-1}(\mu - R_f \mathbf{1}).
\]
Because all investors differ only in $\gamma$, they all hold the same risky portfolio.
\end{frame}

\begin{frame}{Problem A: Step-by-Step Solution (2/2)}
\small
Market clearing implies $w_M$ is proportional to $\Sigma^{-1}(\mu - R_f \mathbf{1})$:
\[
\Sigma w_M = \kappa (\mu - R_f \mathbf{1}).
\]
Take element $i$:
\[
(\Sigma w_M)_i = \kappa (\mu_i - R_f).
\]
But $(\Sigma w_M)_i = \Cov(R_i, R_M)$, so
\[
\mu_i - R_f = \frac{1}{\kappa} \Cov(R_i, R_M).
\]
Apply the same to $M$ to solve for $1/\kappa$, and obtain CAPM.
\end{frame}

\begin{frame}{Problem B: CAPM from an SDF}
\small
\begin{problembox}
Suppose asset prices satisfy $\E[m R_i] = 1$ for all assets, and $m = a - b R_M$.
Show that expected returns obey the CAPM.
\end{problembox}

\textbf{Outline:}
\begin{enumerate}
\item Use $\E[m R_i] = 1$ to express $\E[R_i]$ in terms of $\Cov(m, R_i)$.
\item Substitute $m = a - b R_M$ and simplify.
\item Apply the formula to $R_M$ to identify the slope.
\item Divide the two equations to obtain the CAPM.
\end{enumerate}
\end{frame}

\begin{frame}{Problem B: Step-by-Step Solution}
\small
Using covariance:
\begin{align*}
\E[m R_i] &= \E[m] \E[R_i] + \Cov(m, R_i) = 1, \\
\E[R_i] - R_f &= -\frac{\Cov(m, R_i)}{\E[m]}.
\end{align*}
If $m = a - b R_M$, then $\Cov(m, R_i) = -b \Cov(R_M, R_i)$.
So
\[
\E[R_i] - R_f = \frac{b}{\E[m]} \Cov(R_i, R_M).
\]
Apply the same to $R_M$ and divide to recover CAPM.
\end{frame}

\section{Pitfalls and Extensions}

\begin{frame}{Common Pitfalls}
\small
\begin{itemize}
\item Confusing $R_f$ (gross return) with the risk-free rate (net return).
\item Forgetting to show the market portfolio is efficient.
\item Dropping the key covariance identity $(\Sigma w_M)_i = \Cov(R_i, R_M)$.
\item Mixing up $\beta_i$ with correlations.
\end{itemize}
\end{frame}

\begin{frame}{Extension: Zero-Beta CAPM}
\small
If no risk-free asset exists, the CAPM becomes
\[
\E[R_i] = \E[R_Z] + \beta_i^Z \bigl(\E[R_M] - \E[R_Z]\bigr)
\]
where $R_Z$ is the return on the zero-beta portfolio.

\begin{notebox}
Qualifiers sometimes ask you to derive this formula or interpret $R_Z$.
\end{notebox}
\end{frame}

\begin{frame}{Extension: From CAPM to Factor Models}
\small
CAPM is a one-factor model with factor $R_M$.
Empirical work generalizes this as
\[
\E[R_i] = R_f + \beta_{i1} \lambda_1 + \cdots + \beta_{ik} \lambda_k.
\]

\begin{keypoint}
Many qualifier questions ask whether CAPM is nested in a larger factor model
and how to test that nesting.
\end{keypoint}
\end{frame}

\section{Quick Practice}

\begin{frame}{Two-Minute Drill}
\small
\begin{enumerate}
\item In one sentence, explain why the market portfolio must be mean-variance efficient.
\item Given $\beta_i = 0.7$, $R_f = 1.01$, $\E[R_M] = 1.09$, compute $\E[R_i]$.
\item Explain, in words, why idiosyncratic risk is not priced.
\end{enumerate}
\end{frame}

\begin{frame}{Closing}
\small
We now have two derivations of CAPM that agree line-for-line.
This is the foundation for most empirical asset pricing questions on qualifiers.

\medskip
Next: we will move to testing CAPM and building cross-sectional regressions.
\end{frame}

\end{document}
